\documentclass{article}
\usepackage[utf8]{inputenc}
\usepackage[russian]{babel}   % Подключение модуля для русского языка
\title{Конспект лекций И.А. Халидова по Теории Функции Комплексной переменной}
\author{Григорий Ефимов}
\date{Сентябрь - Декабрь 2021}
\usepackage{mathtools}
\begin{document}

\begin{titlepage}
\maketitle
\end{titlepage}
\tableofcontents
\newpage
\section{Введение}
\subsection{Литература}
\begin{itemize}
	\item Аксенов ТФКП
	\item Волькер Задачник
\end{itemize}
\newpage
\subsection{Как устроены комплексные числа}
\begin{equation}\label{eq:algebraik_form_complex}
z=x+i*y
\end{equation}
\begin{equation}\label{eq:algebraik_form_complex_w}
w=u+i*v
\end{equation}
Формула \ref{eq:algebraik_form_complex} - представление комплексного числа в алгебраическом виде.\\
Z - множество комплексных чисел, т.е. комплексная плоскость. Если задаётся несколько плоскостей, то задаётся буквой W и форумула \ref{eq:algebraik_form_complex_w} соответственно.
\begin{equation}\label{eq:сomplex_module}
|z|=sqrt(x^2+y^2)
\end{equation}
Модуль комплексного числа определяется по формуле \ref{eq:сomplex_module}

\end{document} 
